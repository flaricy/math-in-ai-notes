%----------------------------------------------------------------------------------------
%	PACKAGES AND OTHER DOCUMENT CONFIGURATIONS
%----------------------------------------------------------------------------------------

\documentclass[11pt,fleqn]{book} % Default font size and left-justified equations

\usepackage[top=3cm,bottom=3cm,left=3.2cm,right=3.2cm,headsep=10pt,letterpaper]{geometry} % Page margins
\usepackage{CJKutf8}
\usepackage{xcolor} % Required for specifying colors by name
\definecolor{ocre}{RGB}{52,177,201} % Define the orange color used for highlighting throughout the book

% Font Settings
\usepackage{avant} % Use the Avantgarde font for headings
%\usepackage{times} % Use the Times font for headings
\usepackage{mathptmx} % Use the Adobe Times Roman as the default text font together with math symbols from the Sym­bol, Chancery and Com­puter Modern fonts
\usepackage{microtype} % Slightly tweak font spacing for aesthetics
\usepackage[utf8]{inputenc} % Required for including letters with accents
\usepackage[T1]{fontenc} % Use 8-bit encoding that has 256 glyphs
\usepackage{amsthm}

% Bibliography
\usepackage[style=alphabetic,sorting=nyt,sortcites=true,autopunct=true,babel=hyphen,hyperref=true,abbreviate=false,backref=true,backend=biber]{biblatex}
\addbibresource{bibliography.bib} % BibTeX bibliography file
\defbibheading{bibempty}{}

\input{structure} % Insert the commands.tex file which contains the majority of the structure behind the template

%----------------------------------------------------------------------------------------
%	Definitions of new commands
%----------------------------------------------------------------------------------------

\def\R{\mathbb{R}}
\newcommand{\cvx}{convex}
\begin{document}
\begin{CJK}{UTF8}{gkai} % gbsn 宋体
%----------------------------------------------------------------------------------------
%	TITLE PAGE
%----------------------------------------------------------------------------------------

\begingroup
\thispagestyle{empty}
\AddToShipoutPicture*{\put(0,0){\includegraphics[scale=1.25]{esahubble}}} % Image background
\centering
\vspace*{5cm}
\par\normalfont\fontsize{35}{35}\sffamily\selectfont
\textbf{MATH IN AI WINTER 2023}\\
\vspace*{0.4cm}
{\Huge \textbf{Graph Theory}}\par % Book title
\vspace*{0.4cm}
{\Huge Lecture Notes}\par % Author name
{\Large ymy}\par
\endgroup

%----------------------------------------------------------------------------------------
%	COPYRIGHT PAGE
%----------------------------------------------------------------------------------------

\newpage
~\vfill
\thispagestyle{empty}

%\noindent Copyright \copyright 2014 Andrea Hidalgo\\ % Copyright notice

\noindent \textsc{Personal use}\\

\noindent {https://github.com/flaricy/notes-for-graph-theory}\\ % URL

\noindent The author hopes to take notes while learning graph theory. Reference books are \textit{Algebraic Graph Theory} and 离散数学基础. Starts from Dec 8th. 

\noindent \textit{Not released yet} % Printing/edition date

%----------------------------------------------------------------------------------------
%	TABLE OF CONTENTS
%----------------------------------------------------------------------------------------

\chapterimage{head1.png} % Table of contents heading image

\pagestyle{empty} % No headers

\tableofcontents % Print the table of contents itself

%\cleardoublepage % Forces the first chapter to start on an odd page so it's on the right

\pagestyle{fancy} % Print headers again

\newpage
\thispagestyle{empty}
\centering 
\vspace*{10cm}
\textit{This page is intentionally left blank.}
%----------------------------------------------------------------------------------------
%	CHAPTER 1
%----------------------------------------------------------------------------------------

\chapterimage{head2.png} % Chapter heading image
\chapter{Convex Sets}
\section{Convexity}
\subsection{Cone}
\begin{definition}[Cone]
A set $K \in \R^n$, when $x \in K $ implies $\alpha x \in K$.
\end{definition}
A non convex cone can be hyper-plane.\\
For convex cone $x + y \in K, \forall x,y \in K$.\\
Cone don't need to be "pointed". e.g. \\
Direct sums of cones $C_1 + C_2 = \{ x = x_1+x_2 | x_1 \in C_1, x_2 \in C_2 \}$.\\
\begin{example}
$S_1^n  \{ X | X=X^n ,\lambda(x) \ge 0\}$\\
A matrix with positive eigenvalues.
\end{example}

\subsubsection{Operations preserving convexity}
\begin{itemize}
\item[Intersection] $C  \cap_{i \in \mathbb{I}}C_i$
\item[Linear map] Let $A : \mathbb{R}^n \to  \R^n$ be a linear map. If $C \in \R^n$ is convex, so is $A(C) = \{Ax \forall x \in C \}$
\item[Inverse image] $A^{-1}(D) = \{ x \in \R |Ax \in D \}$
\end{itemize}

\subsubsection{Operations that induce convexity}
Convex hull on $S = \cap \{C | S\in C, C is convex\}$\\
\begin{example}
$Co \{ x_1,x_2,\cdots,x_m\} = \{ \sum_{i=1}^m \alpha_i x_i | \alpha \in \delta_m \}$
\end{example}
For a convex set $x \in C \Rightarrow x = \sum \alpha_i x_i$. 
\begin{theorem}[Carathéodory's theorem]
If a point $x \in \R^d$ lies in the convex hull of a set $P$, there is a subset $P'$ of $P$ consisting of $d + 1$ or fewer points such that $x$ lies in the convex hull of $P'$. Equivalently, x lies in an r-simplex with vertices in P.
\end{theorem}

\section{Convex Functions}
\begin{definition}[Convex function]
Let $C \in \R^n$ be convex, $f:C \to \R$ is convex on f if $x,y \in C \times C$. $\forall \alpha \in (0,1)$, $f(\alpha x + (1-\alpha) y) \le f(\alpha x) + f((1-\alpha) y)$
\end{definition}

\begin{definition}[Strictly Convex function]
Let $C \in \R^n$ be convex, $f:C \to \R$ is strictly convex on f if $x,y \in C \times C$. $\forall \alpha \in (0,1)$, $f(\alpha x + (1-\alpha) y) \langle f(\alpha x) + f((1-\alpha) y)$
\end{definition}

\begin{definition}[Strongly convex]
$f:C \to \R$ is strongly convex with modules $u \ge 0$ if $f - \frac{1}{2}u || \cdot ||^2$ is convex.
\end{definition}
Interpretation: There is a convex quadratic $\frac{1}{2}u || \cdot ||^2$ that lower bounds f.
\begin{example}
$\min_{x \in C} f(x) \leftrightarrow \min \bar{f}(x)$
Useful to turn this into an unconstrained problem. \\
$$\bar{f}(x) = \begin{cases}
f(x) \quad if x \in C \\
\infty \quad  elsewhere
\end{cases}$$
\end{example}
\begin{definition}
A function $f : \R^n \to \R \cup \infty \ \bar{\R}$ is convex if $x,y \in \R^n \times \R^n$, $\forall x,y , \bar{f}(\alpha x + (1-\alpha) y) \le f(\alpha x) + f((1-\alpha) y)$
\end{definition}
Definition 1 is equivalent to definition 2 if $f(x) = \infty$.
\begin{example}
$f(x) = \sup_{j \in J} f_j(x)$
\end{example}

\subsection{Epigraph} 
\begin{definition}[Epigraph]
For $f: \R^n \rightarrow \bar{R}$, its epigraph $epi(f) \in \R^{n+1} is the set epi(f) \{ (x,\alpha) | f(x) \in \alpha \}$
\end{definition}
Next: a function is convex i.f.f. its epigraph is convex.

\begin{definition}
A function $f : C \rightarrow \R, C \in \R^n$ is convex if $\forall x, y \in C$, $f(ax + (1-a)x) \le af(x) + (1-a)f(x) \quad \forall a \in (0,1)$.\\ 
Strict convex: $x \neq y \Rightarrow f(ax + (1-a)x) \le af(x) + (1-a)f(x) $
\end{definition}
\begin{remark}
$f$ is convex $\Rightarrow$ $-f$ is concave.
\end{remark}
Level set: $S_{\alpha}f = \{ x | f(x) \le \alpha \}$.\\ 
$S_{\alpha}f$ is convex $\Leftrightarrow$ $f$ is convex. \\
\begin{definition}[Strongly convex]
$f : C \rightarrow \R$ is strongly convex with modules $\mu$ if $\forall x, y \in C$, $\forall \alpha \in (0,1)$, $f(ax + (1-a)x) \le af(x) + (1-a)f(x) - \frac{1}{2\mu}\alpha(1- \alpha) \|x-y\|^2$.
\end{definition}

\begin{remark}
\begin{itemize}
\item $f$ is 2nd-differentiable, $f$ ix \cvx $\iff$ $\nabla^2f(x) \rangle  0$.
\item $f$ is strongly \cvx $\iff$ $\nabla^2f(x) \rangle  \mu I$ $\iff$ $x \ge \mu$
\end{itemize}
\end{remark}
\begin{definition}[2]
$f : \R^n \to \bar{\R} $ is \cvx  if $x, y  \in \R , \alpha \in (0,1), f(ax + (1-a)x) \le af(x) + (1-a)f(x)$.  
\end{definition}
The effective domain of $f$ is $dom f = \{x | f(x) \langle + \infty \}$ 
\begin{example}[ludcator function]
$\delta_c(x) = \begin{cases}
0 \quad  x \in C \\
+ \infty \quad elsewhere
\end{cases}$.\\
$dom \space \delta_c(x) = C$
\end{example}
\begin{definition}[Epigraph]
The epigraph of f is $epi \space f = \{(x,\alpha) | f(x) \le \alpha\}$
\end{definition}
The graph of $epi \space f$ is $\{ (x,f(x) | x \in dom \space f\}$.
\begin{definition}[III]
A function $f : \R^n \to \bar{\R}$ is %\cvx  if $\epi \space f $ is \cvx
\end{definition}
\begin{theorem}
$f : \R^n \to \bar{\R}$ is \cvx  $\iff$ $\forall x,y \in \R^n, \alpha \in (0,1), f(ax + (1-a)x) \le af(x) + (1-a)f(x)$.
\end{theorem}
\begin{proof}
$\Rightarrow$ take $x,y \in dom \space f$, $(x,f(x)) \in epi \space f$,$(y,f(y)) \in epi \space f$.
\end{proof}

\begin{example}[Distance]
Distance to a \cvx  set $d_c(x) = \inf \{ \| z-x \| | z \in C \}$. Take any two sequence $\{ y_k\} and \{ \bar{y}_k\} \subset C$ s.t. $\| y_k - x\| \to d_c(x)$, $\| \bar{y}_k - \bar{x}\| \to d_c(\bar{x})$. $z_k = \alpha y_k + (1 - \alpha) \bar{y}_k$.
\begin{align*}
d_c(\alpha x + (1-\alpha) \bar{x}) &\le \| z_k - \alpha x - (1 - \alpha) \bar{x}\| \\
& = \| \alpha(y_k - x) + (1 - \alpha)(\bar{y}_k - \bar{x})\| \\
& \le \alpha \| y_k - x\| + (1 - \alpha ) \|\bar{y}_k - \bar{x}\|
\end{align*}
Take $k \to \infty$, $d_c(\alpha x + (1 - \alpha) \bar{x}) \le \alpha d(x) + (1 - \alpha) d(\bar{x})$
\end{example}
\begin{example}[Eigenvalues]
Let $X \in S^n := \{ n \times n symmetric matrix\}$. $\lambda_1(x) \ge \lambda_2(X) \ge \ldots \ge \lambda_n(x)$.\\
$f_k(x) = \sum_{1}^n \lambda_i(x)$.\\
Equivalent characterization 

\begin{align*}
f_k(x) & = \max\{ \sum_{i} v_i^T Xv_i | v_i \perp v_j , i \neq j\} \\
& =  \max\{ tr( V^TXV | V^T V = I_k \} \\
\max \{tr(VV^TX) \} \text{by circularity}
\end{align*}
Note $\langle A,B\rangle  = tr(A,B)$ is true for symmetric matrix. \\
$\langle A,A\rangle  = |A |_F^2 = \sum_{i} A_{ii}^2$
\end{example}

\section{Support Function}
Take a set $C \in \R^n$, not necessarily convex.The support function is $\sigma_C = \R^n \to \bar{\R}$. $\sigma_C(x) = \sum \{ \langle x,u\rangle  | u \in C\}$.
\includegraphics[scale=0.5]{1_1.png}
\begin{fact}
The support function binds the supporting hyper-plane.
\end{fact}

Supporting functions are
\begin{itemize}
\item Positively homogeneous\\
$\sigma_C(\alpha x) = \alpha \sigma_C(x) \forall \alpha \rangle  0$ \\
$\sigma_C(\alpha x ) = \sup_{u \in C} \langle \alpha x, u\rangle  = \alpha \sup_{u \in C} \langle x, u\rangle  = \alpha \sigma_C(x)$
\item Sub-linear( a special case of convex, linear combination holds $\forall \alpha$.\\
$\sigma_C(\alpha x + (1 - \alpha) y ) = \sup_{u \in C} \langle \alpha x + (1 - \alpha) y,u\rangle  \le \alpha\sup_{u \in C}\langle x,u\rangle  + (1 - \alpha)\sup_{u \in C}\langle y,u\rangle  $
\end{itemize}
\begin{example}[L2-norm]
$\| x \| = \sup_{u \in C} \{ \langle x, u \rangle, u \in \R^n \}$.\\
$\|x \|_p = \sup \{ \langle x, u \rangle, u \in B_q \}$ where $\frac{1}{p} + \frac{1}{q} = 1$. $B_q = \{ \|x \|_q \le 1\}$.\\
The norm is 
\begin{itemize}
\item Positive homogeneous
\item sub-linear
\item If $0 \in C$, $\sigma_C$ is non-negative.
\item If $C$ is central-symmetric, $\sigma_C(0) = 0$ and $\sigma_C(x) = \sigma_C(-x)$
\end{itemize}
\end{example}

\begin{fact}[Epigraph of a support function]
$epi \space \sigma_C = \{ (x,t) | \sigma_C(x) \le t\}$.
Suppose $(x,t) \in epi \space \sigma_C$. Take any  $\alpha > 0$. $\alpha(x,t) = (\alpha x, \alpha t)$.\\
$\alpha \sigma_C(x) = \alpha \sigma_C(x) \le \alpha t$. $\alpha(x,c) \in epi 
\sigma_C$\\
\includegraphics[]{1_2}
\end{fact}

\section{Operations Preserve Convexity of Functions}
\begin{itemize}
\item Positive affine transformation \\
$f_1,f_2,\ldots,f_k \in \space cvx \R^n$.\\
$f = \alpha_1 f_1 + \alpha_2 f_2 + \ldots + \alpha_k f_k$
\item Supremum of functions. Let $\{ f_i \}_{i \in I}$ be arbitrary family of functions. If $\exists x \sup_{j \in J} f_j(x) < \infty \Leftrightarrow f(x) = \sup_{j \in J} f_j(x) $\\
\includegraphics[]{1_3}
\item Composition with linear map.\\
$f \in cvx \R^n$, $A:\R^n \to \R^m$ is a linear map.
$f \circ A (x) = f(Ax) \in cvx \R^n$\\
\begin{align*}
f \circ A (x) & = f(A(\alpha x + (1-\alpha) y)) \\
& = f(A \alpha x + (1-\alpha) A y) \\
& \le \alpha f(Ax) + (a - \alpha) f(Ay)
\end{align*}
\end{itemize}

% chapter 2: Basic concepts
\chapterimage{head2.png} % Chapter heading image
\chapter{Basic Concepts}

% section 2.1
\section{Simple graph, complete graph, tournament graph}
\begin{definition}
[simple graph] 有向图或者无向图,如果无平行边(重边)和自环。
\end{definition}

% section 2.2
\section{Operations on a graph}
点、边的删除;收缩;两个图之间的运算

% section 2.3
\section{Havel-Hakimi algorithm \& Erdos-Gallai theorem}
给定一个度数序列 $\{d_i\}$,判断是否可以根据这个度数序列构造出简单无向图。
\begin{theorem}
    [Havel-Hakimi] Let $d = (d_1, d_2, ...,d_n)$, $\sum_{i = 1}^n {d_i} = 0 (mod 2)$ 且 $n-1 \geq d_1 \geq 
    d_2 \geq ... \geq d_n \geq 0$, then $d$ 简单可图化 $\iff$ $d' = (d_2 - 1, d_3 - 1, ... d_{d_1 + 1} - 1, ...)$ 简单可图化。 
\end{theorem}
\begin{proof}
    \textit{to be written}
\end{proof}

\begin{theorem}
    [Erdos-Gallai] 设 $d = (d_1, d_2 ,... d_n)$ 满足 $n-1 \geq d_1 \geq 
    d_2 \geq ... \geq d_n \geq 0$, 则 $d$ 简单可图化 $\iff$ $\forall 1 \leq r \leq n - 1$, 
    \[
        (1)\sum_{i = 1}^r d_i \leq r(r -1) + \sum_{i = r + 1}^{n} \min\{r, d_i\} \newline (2) \sum_{i = 1}^{n} d_i = 0 (mod 2)
        \]
\end{theorem}
\section{通路与回路}

\begin{definition}[通路]
    设G为无向标定图,G中顶点与边的交替序列 $\Gamma = v_{i_0}e_{j_1}v_{i_1}e_{j_2}...e_{j_l}v_{i_l}$ 称为
    顶点 $v_{i_0}$ 到 $v_{i_l}$ 的 \textbf{通路}。
\end{definition}

\begin{definition}
    [walk 简单通路]
    若$\Gamma$中所有边各异。
\end{definition}
\begin{definition}
    [closed walk 简单回路]
    若$\Gamma$是简单通路且 $v_{i_0} = v_{i_l}$
\end{definition}

\begin{definition}
    [path 初级通路,路径]
    如果$\Gamma$的所有顶点各异,所有边各异。
\end{definition}

\begin{definition}
  [初级回路,圈] 
  若$\Gamma$是初级通路且起始点 = 终点。   
\end{definition}

\begin{remark}
    \begin{itemize}
        \item 初级通路是简单通路。
        \item 上述定义针对的是无向图。有向图类似。
    \end{itemize}
    
\end{remark}
\begin{definition}
    [Girth] 围长。简单无向图中最短圈的长度。
\end{definition}
\begin{definition}
    [Perimeter] 周长。简单无向图中最长圈的长度。
\end{definition}

\chapterimage{head2.png} % Chapter heading image
\chapter{Eulerian Graph and Hamiltonian Graph}
\section{Eulerian Graph}
\begin{definition}
    [Eulerian trail/path 欧拉通路] an Eulerian trail (or Eulerian path) is a trail in a finite graph that visits every edge exactly once (allowing for revisiting vertices).
\end{definition}
\begin{definition}
    [Eulerian circuit/cycle] an Eulerian trail that starts and ends on the same vertex
\end{definition}

\begin{definition}
    [Eulerian Graph] 具有欧拉回路的图。规定平凡图为欧拉图。
\end{definition}
\begin{definition}
    [半欧拉图] 具有欧拉通路但没有欧拉回路的图。
\end{definition}

\begin{lemma}
    如果G是欧拉图,那么从G的任何一个顶点出发都可以找到一条简单通路构成欧拉回路。
\end{lemma}
\begin{proof}
    由定义验证即可。
\end{proof}

\begin{theorem}
    Suppose G is a indirected graph. The following 3 propositions are equivalent: \\
    (1) G is Eulerian \\
    (2) The degree of every vertex in G are even. And G is connected.\\ 
    (3) G is the union of several cycles with no intersected edges. And G is connected.

\end{theorem}
\begin{proof} 
    (1) $\implies$ (2) \\
    如果 G 平凡,结论显然。否则结论也显然。 \\
    (2) $\implies$ (3) \\
    我们证明: G的任意顶点度数为偶数 $\iff$ G是若干个边不交的权的并。 对边数归纳: \\
    如果边数 = 2, 结论显然。 假设$2(m-1)$条边时成立,当$2m$条边时。找一个连通分支$G' = <V', E'>$, 设其中的一条极大路径为$v_1,v_2,...,v_n$,由于$d(v_1) \geq 2$,
    从而存在 $v_i(i \geq 2)$,使得$v_1$ 和 $v_i$ 相邻,从而找到一个简单回路$L$。从$G'$中删去$L$,即可归纳。\\
    (3) $\implies$ (1) \\
    对圈的个数归纳。\\
    一个圈时显然成立。
    假设m-1个圈时成立,则$m\geq 2$个圈时:先任意选一个圈$L_1$,由于图连通,从而$\exists v_1, v_2 \in L_1, L_2$, $v_1, v_2$可以是同一个点。 由归纳假设和Lemma 1,可以从$v1$出发在剩下的圈中走到$v2$,最后绕$L_1$这个圈。

\end{proof}

\begin{theorem}
    [半欧拉图的判定]
    设G为连通的无向图,则G是半欧拉图当且仅当G中恰好有两个奇度顶点。
\end{theorem}
\begin{proof}
    在这两个顶点之间连一条边。由Theorem 3.1.2知存在欧拉回路,然后在回路中去掉那条添加的边即可。\\
    假设由欧拉回路,那么由Theorem 3.1.2知没有奇数度的顶点。矛盾。
\end{proof}

\begin{corollary}
    设G为连通的无向图,G中有$2k$个奇度顶点,则G中存在k条边不交的简单通路$P_1,P_2,...P_k$,使得$E(G) = \cup_{i = 1}^nE(P_i)$.
\end{corollary}
\begin{proof}
    将2k个点两两配对,每对连一条边。在新图中取一个欧拉回路,然后删去这k条边,一定能得到k段不交的简单通路。    
\end{proof}

\begin{theorem}
    [有向欧拉图的判定]
    设D为有向图,则下面三个命题等价: \\
    (1) D is Eulerian. \\
    (2) D is connected and $\forall v \in V(D), d^+(v) = d^-(v)$. \\
    (3) D is connected and is the union of several indirected cycles with no intersected edges.
\end{theorem}
\begin{proof}
    (1) $\implies$ (2) : trival \\
    (2) $\implies$ (3) : similar to Theorem 8.1.1 \\
    (3) $\implies$ (1) : similar to Theorem 8.1.1 \\
\end{proof}

\begin{theorem}
    [有向半欧拉图的判定]
    D中恰有两个奇数度的顶点,且一个点出度比如度大1,一个点入度比出度大1.其他点如度都等于出度。
\end{theorem}

\begin{definition}
    [Fleury's algorithm]
    求无向图中的欧拉回路。
\end{definition}

\begin{definition}
    [逐步插入回路算法]
\end{definition}

\section{Hamiltonian Graph}
\begin{proposition}
    [motive] (1859 Willian Hamilton) \textbf{Traverse the World Problem} \\
    正十二面体图上是否存在一条初级回路(圈)遍历所有顶点?
\end{proposition}
\begin{remark}
    [Traveling Salesman Problem, TSP] 在一个赋权的无向图中,去找一个哈密尔顿回路,并且使得该回路的总权值最小。\\
    这是一个NP-complete问题。

\end{remark}

\begin{definition}
    Hamiltonian path \\
    \begin{itemize}
        \item 经过所有点恰好1次的通路称为哈密顿路。
        \item 如果上面的通路是回路,那么称为哈密顿回路。
        \item 具有哈密顿回路的图称为哈密顿图。
        \item 具有哈密顿通路但没有哈密顿回路的图称为半哈密顿图。
        \item 规定:平凡图是哈密顿图。
    \end{itemize}
\end{definition}

\begin{theorem}
    $G = < V, E>$ is Hamiltonian. For every $\emptyset \neq V_1 \subsetneq{V}$we have
    \[
        p(G - V_1) \leq |V_1| \], where $p(G)$ denotes the number of connected components of $G$.
\end{theorem}
\begin{proof}
    [\textbf{Sketch}]
    \begin{fact}
        往一个图中添加边,那么连通分支数不会减少。
    \end{fact}
    所以只需要考虑$G$中哈密顿回路所包含的边,如果删去一些节点,对连通分支数的影响。
\end{proof}

\begin{corollary}
    [case of semi-Hamiltonian graph] 设$G = < V, E>$ is semi-Hamiltonian, For every $\emptyset \neq V_1 \subsetneq{V}$ we have
    \[
        p(G - V_1) \leq |V_1| + 1 \], where $p(G)$ denotes the number of connected components of $G$.
\end{corollary}
\begin{proof}
    同样考虑在哈密顿通路(不是回路)中删除一些点能产生多少段即可。
\end{proof}

\begin{example}
    [Peterson Graph] Show that Peterson Graph is semi-Hamiltonian.
\end{example}
\begin{remark}
    彼得森图满足Theorem 3.2.2,但不是哈密顿图。
\end{remark}

\begin{theorem}
    [a sufficent condition] 设G为$n \geq 1$ 阶简单无向图,若对于$G$中不相邻的任意两点$v_1,v_2$,均有
    \[
        d(v_1) + d(v_2) \geq n  - 1\]
    则G中存在哈密顿通路。
\end{theorem}
\begin{proof}
    First, show that $G$ is connected. \\
    if G is not connected, then consider two connected components $G_1, G_2$, pick $v_1 \in G_1, v_2 \in G_2$.
    We have $d(v_1) \leq |V(G_1)| - 1, d(v_2) \leq |V(G_2)| - 1$, which implies $d(v_1) + d(v_2) \leq |V| - 2$. Contradiction. \\
    考虑极大路径法。\\
    1. 任选一条极大路径 $\Gamma = v_1 v_2 ... v_l$。 如果 $l = n$, 那么这就是哈密顿通路。如果不是: \\
    2. 证明存在一个圈经过$\Gamma$ 上所有顶点。为此,只要证明:存在顶点 $v_i \in \Gamma$ 使得 $v_{i-1}$ 与 $v_1$ 相邻,
    $v_i$ 与 $v_n$ 相邻。这一点可以通过条件证明。 \\ 
    因为图G中有$\Gamma$中未出现的点,所以可以找到这样一个点和上面构造的圈中某个点相邻,我们就可以找到一条更长的哈密顿通路。\\
    3. 重复此过程,我们可以在有限步之内得到最长的极大通路,此即哈密顿通路。
\end{proof}

\begin{corollary}
    [Øystein Ore, Norwegian] 设G为$n \geq 1$ 阶简单无向图,若对于$G$中不相邻的任意两点$v_1,v_2$,均有
    \[
        d(v_1) + d(v_2) \geq n\]
    则G中存在哈密顿通路。
\end{corollary}
\begin{proof}
    由Theorem3.2.5知,G中存在哈密顿通路$\Gamma = v_1 v_2 ... v_n$,如果$v_1 和 v_n$相邻,那么找到了回路。如果不相邻,用类似的方法可以找到一个圈。    
\end{proof}

\begin{corollary}
    设G为$n \geq 1$ 阶简单无向图,若对于$G$中任意$v$,均有
    \[
        d(v)\geq \frac n 2\]
    则G中存在哈密顿通路。
\end{corollary}

\begin{theorem}
    设$u,v$为无向n阶简单图G中的任意两个不相邻的顶点,且$d(u) + d(v) \geq n$,则 \\
    G为哈密顿图 $\iff$ $G \ \cup \ e = (u, v)$ 为哈密顿图。
\end{theorem}
\begin{proof}
    $\implies$ trivial. \\
    $\impliedby$ 如果$G \ \cup \ e = (u, v)$ 的一条哈密顿回路中有边 $e$,那么删去这条边,用同样的方法构造一个圈。
\end{proof}

\begin{example}
    对于 $n \geq 4$阶简单无向图G,只要$\delta(G) \geq \frac n 2 + 1$,G中至少存在2条不同的哈密顿回路。
\end{example}

\begin{theorem}
    设$D$为$n \geq 2$阶竞赛图,则D具有哈密顿通路。
\end{theorem}
\begin{proof}
    induction by n. \\
    n = 2 : trivial \\
    Suppose n = k OK. When n = k + 1. WLOG, let $\Gamma = v_1v_2...v_k$ be a Hamiltonian path of $G - v_{k+1}$,
    If $\forall 1\leq i \leq k$, directed edge $(v_i, v_{k+1}) \in E(G)$, then $\Gamma' = v_1...v_kv_{k+1}$ is the desired Hamiltonian path. 
    Otherwise, there $\exists r \in \{1,2, ... ,k\}$ such that $(v_i, v_{k+1}) \in E(G), \forall i < r$, but $(v_{k+1}, v_r) \in E(G)$. Then $\Gamma' = v_1...v_{r-1}v_{k+1}v_r...v_k$ 
    is the desired path.   
\end{proof}

下面探讨竞赛图中何时有哈密顿回路。我们假定竞赛图是强连通的,那么有如下的两个引理。

\begin{lemma}
    强连通的竞赛图$(n\geq 3)$中存在长度为3的圈。
\end{lemma}
\begin{proof}
    Take any vertex $v_0 \in D$. Let $\Gamma_D^+(v_0) = \{v|<v_0, v> \in E(D)\}$, $\Gamma_D^-(v_0) = \{v|<, v_0> \in E(D)\}$.
    We claim that both $\Gamma^+$ and $\Gamma^-$ are non-empty. Moreover, there $\exists u \in \Gamma^+, v \in \Gamma^-$ such that
    $(u, v) \in E(D)$ (otherwise D is not strongly connected). Thus, $v_0 \to u \to v \to v_0$ forms a 3-loop.
\end{proof}

\begin{lemma}
    强连通的竞赛图$(n\geq 3)$中,如果存在长为$k < n$的圈,则存在长为$k+1$的圈。
\end{lemma}
\begin{proof}
    1. 先考虑是否有点$u$同时是长为k的圈有关的边的from和to。如果是,那么必然存在$(v_i,u) \in E \wedge (u, v_{i+1}) \in E$, 这一点用之前的套路即可说明。\\
    2. 如果不存在这样的点,那么圈以外的点可以分成两类,一类是圈上的点到该点都是入边,一类是圈上的点到该点都是出边。\\
    由强连通性可知,这两个集合都非空,且任何一个都有指向对方集合的边。那么任意选出圈上的点$v_1 \to v_2\to v_3$,将其换成$v_1 \to v' \to v'' \to v_3$.
\end{proof}

\begin{theorem}
    强连通的竞赛图是哈密顿图。
\end{theorem}
\begin{proof}
    Easy by the previous 2 lemmas. \\
    Note that $K_2$ cannot be strongly connected but other tournament graphs can.
\end{proof}

\begin{theorem}
    $K_{2n}$中有n-1条边不重的哈密顿回路,$K_{2n+1}$中有n条边不重的哈密顿回路。
\end{theorem}

% chapter: matrix theory for graphs
\chapterimage{head2.png} % Chapter heading image
\chapter{Matrix Theory for Graphs}
\textbf{以下提到的graph指的是无向图,简单起见可以认为没有自环。允许有重边。}
\section{Then Adjacency Matrix}
\begin{definition}
    [adjacency matrix of a directed graph 邻接矩阵] The adjacency matrix $A(X)$ of a directed graph $X$ is the interger matrix with rows and columns indexed by the vertices of $X$, such that the $uv$-entry of $A(X)$
    is equal to the number of arcs from u to v. If $X$ is simple, then the elements are 0 or 1.
\end{definition}

\begin{definition}
    [adjacency matrix of an undirected graph] We view each edge as a pair of arcs in opposite directions, and $A(X)$ is a symmetric 01-matrix. If the graph has no loops (自环), the diagonal entries of $A(X)$ are 0.
\end{definition}

\begin{remark}
    注意是这里的矩阵应该是标定的(顶点指定编号)。同一个顶点集如果采用不同的方式,那么得到的邻接矩阵不同。但是,这些矩阵之间存在关联。具体而言:
\end{remark}

\begin{lemma}
    Let X and Y be directed graphs on the same vertex set. Then they are isomorphic if and only if there is a permutation matrix P such that $P^TA(X)P = A(Y)$.
\end{lemma}
\begin{proof}
    将邻接矩阵视为双线性函数的度量矩阵。在不同的基下双线性函数的度量矩阵是合同的。在此处基的变换矩阵是permutation matrix.
\end{proof}
\begin{remark}
    Since permutation matrices are orthogonal, i.e. $P^T = P^{-1}$, and so if X and Y are isomorphic, then $A(X)$ and $A(Y)$ are similar matrices.
\end{remark}
\begin{definition}
    [the spectrum of a matrix 矩阵的谱] The \textit{spectrum} of a matrix is the list of its eigenvalues together with their multiplicities.
\end{definition}
\begin{definition}
    [the spectrum of a graph] The spectrum of a graph $X$ is the spectrum of $A(X)$. 我们也称$A(X)$的特征值和特征向量是图$X$的特征值和特征向量.
\end{definition}
\begin{remark}
    Lemma 4.1.1 shows that the spectrum (or equivalently, the characteristic polynomial of $A(X)$) is an invariant of the isomorphism class of a graph. \\
    但是两个不同构的图也可以有相同的特征多项式,
\end{remark}
下面探讨邻接矩阵提供的更多信息。

\begin{definition}
    [walk 简单通路] A walk of length $r$ in a directed graph $X$ is a sequence of vertices
    \[
        v_0 \sim v_1 \sim ... \sim v_r\]
    A walk is \textit{closed} if $v_0 = v_r$.
\end{definition}
\begin{remark}
    注意与path区分。path不允许顶点重复。
\end{remark}

\begin{lemma}
    Let $X$ be a directed graph with adjacency matrix $A$. The number of walks from $u$ to $v$ in $X$ with length $r$ is $(A^r)_{uv}$.
\end{lemma}
\begin{proof}
    Induction by n. Consider the meaning of matrix multiplication.
\end{proof}
\begin{remark}
    Lemma 4.1.2 shows that the number of closed walks of length $r$ in X is $tr(A^r)$.
\end{remark}

\begin{corollary}
    Let $X$ be a graph (无自环的无向图) with $e$ edges and $t$ triangles. A : adjacency matrix, then \\
    (1) tr $A$ = 0 \\
    (2) tr $A^2$ = $2e$ \\
    (3) tr $A^r$ = $6t$ 
\end{corollary}

\section{The Incidence Matrix}
\begin{definition}
    [incidence matrix of an undirected graph 无向图的关联矩阵] Let $G = <V, E>$, $V = \{v_1, ... ,v_n\}$, $E = \{e_1, ...,e_m\}$.
    The incidence matrix $B(X) \in M_{n\times m}(\mathbb{Z})$, such that
    \[
        b_{ij} = \begin{cases}
            1, \quad if \ v_i \in e_j \\
            0, \quad otherwise
        \end{cases}\]
\end{definition}

\begin{theorem}
    Let X be a graph with $n$ vertices and $c_0$ bipartite connected components 连通的二部图分量. Then rk $B = n - c_0$.
\end{theorem}
\begin{proof}
    We shall show that the null space of $B$ has dimension $c_0$.    
\end{proof}

\begin{lemma}
    Let $B$ be the incidence matrix of the graph X. Then $BB^T = \Delta(X) + A(X)$, where 
    \[
        \Delta(X) = diag\{deg(v_1), ... , deg(v_n)\}\]   
\end{lemma}
\begin{proof}
        比较平凡。    
\end{proof}

% 这是一个外国参考文献中没有的概念
\begin{definition}
    [fundamental incidence matrix 基本关联矩阵] 设$v_i$为参考点,在$B(X)$中删去第$i$行得到的矩阵。
\end{definition}

% to be written: 关于rank和连通性的定理

\section{The Incidence Matrix of an Oriented Graph} 
可以理解为有向图。 \\
An \textit{orientation} of a (无向图)graph $X$ is the assignment of a direction to each edge.\\
Recall that an \textbf{arc} of a graph is an ordered pair of adjacent vertices $(u, v)$. To put the definition of orientation formally,
\begin{definition}
    [orientation of a graph] a function $\sigma : E \to \{-1, 1\}$, where $E$ is the arcs of X. The function satisfies that if $(u, v)$ is an arc, then 
    \[
        \sigma (u,v) = -\sigma(v, u).\]
    If $\sigma (u,v) = 1$, then we will regard the edge $uv$ as $u \to v$.
\end{definition}

Now an \textit{Oriented Graph} is a graph together with a particular orientation. We may use $X^\sigma$ to denote it.
\begin{definition}
    [incidence matrix of a directed graph] $D(X^\sigma)$ is the $\{0, 1, -1\}$-matrix with rows indexed by vertices, columns indexed by edges of X. 
    \[
        d_{uf} = \begin{cases}
            -1, \quad &u \to \_ \\
            1, \quad  &\_ \to u \\
            0, \quad & else 
        \end{cases}\]
\end{definition}
\begin{remark}
    注意这个和离散数学书上的定义是反的。
\end{remark}

\begin{theorem}
    If $X$ has $c$ connected components. Let $\sigma$ be any orientation of $X$ and $D$ is the incidence matrix of $X^\sigma$, then rk $D = n - c$. 
\end{theorem}
\begin{proof}
    Show that the null space of D has dimension $c$. Suppose $z \in \R^n$ such that $z^TD = 0$. 根据定义验证每一个连通分支中的$z_i$应该取相同值。
\end{proof}
% new chapter : matrix
\def\x{\times}

\begin{theorem}
    $DD^T = \Delta(X) - A(X)$.
\end{theorem}
\begin{proof}
    to be written.
\end{proof}

\chapterimage{head1.png}
\chapter{Matrix Foundations}

\section{线性映射的伴随}

\section{Singular Value Decomposition}
设$V, W$为$\R$上的有限维内积空间,其内积记作$(\cdot | \cdot)_V$ 和 $(\cdot | \cdot)_W$. 设 $m = \text{dim} W$, $n = \text{dim} W$.
\begin{theorem}
    [SVD] 对于任意线性映射 $T: V \to W$, 记$p = \min \{m, n\}$, 则存在 \\
    \begin{itemize}
        \item V 的单位正交基$v_1,...,v_m$, \\
        \item W 的单位正交基$w_1, ..., w_m$, \\
        \item 非负实数$\sigma_1 \geq \sigma_2 \geq ... \geq \sigma_p$, 
    \end{itemize}
    使得
    \[
        Tv_i = \begin{cases}
            \sigma_iw_i, \quad &1 \leq i \leq p \\
            0, \quad &i > p 
        \end{cases}\]
    此处的$\sigma_1 \geq ... \geq \sigma_p$ 由$T$唯一确定,称为$T$的奇异值。
\end{theorem}

\begin{proof}
    to be written
\end{proof}

\begin{theorem}
    [矩阵版本] 设 $V = \R^m, W = \R^n$, 各自配备标准内积,并且将$T$等同于矩阵$A \in M_{n \times m}(\R)$. 对于定理5.1.1中的单位正交基,以列向量定义正交矩阵:
    \[
        P := (v_1|...|v_m) \in M_{m \times m}(\R), \; Q := (w_1|...|w_n) \in M_{n \times n} (\R),\]
    再用奇异值定义
    \[
        \Sigma := \text{diag}\{\sigma_1, \sigma_2, ... \sigma_r, 0, 0...\} \in M_{n\times m}(\R)\]
        其中 $r = \text{rank}(T)$.
    则奇异值分解化为矩阵等式
    \[
        AP = Q\Sigma\]
    亦即 
    \[
        A = Q \Sigma P^T \]
\end{theorem}


\section{Kronecker Product}
\begin{definition}
    设 $A \in M_{n \x m }(\R), B \in M_{p \x q}(\R)$, 则定义
    \[
        A \otimes B = 
        \begin{pmatrix}
            a_{11}B & a_{12}B & ... & a_{1n}B \\
            a_{21}B & a_{22}B & ... & a_{2n}B \\
            ... \\
            a_{m1}B & a_{m2}B & ... & a_{mn}B 
        \end{pmatrix}\]
\end{definition}
\begin{property}
    [混合乘积] \[
        (A\otimes B)(C \otimes D) = (AC) \otimes (BD)\]
\end{property}
\begin{proof}
    利用分块矩阵乘法即可验证。
\end{proof}

\end{CJK}
\end{document}